\documentclass[journal,12pt,twocolumn]{IEEEtran}
%
\usepackage{setspace}
\usepackage{gensymb}
%\doublespacing
\singlespacing

%\usepackage{graphicx}
%\usepackage{amssymb}
%\usepackage{relsize}
\usepackage[cmex10]{amsmath}
%\usepackage{amsthm}
%\interdisplaylinepenalty=2500
%\savesymbol{iint}
%\usepackage{txfonts}
%\restoresymbol{TXF}{iint}
%\usepackage{wasysym}
\usepackage{amsthm}
%\usepackage{iithtlc}
\usepackage{mathrsfs}
\usepackage{txfonts}
\usepackage{stfloats}
\usepackage{bm}
\usepackage{cite}
\usepackage{cases}
\usepackage{subfig}
%\usepackage{xtab}
\usepackage{longtable}
\usepackage{multirow}
%\usepackage{algorithm}
%\usepackage{algpseudocode}
\usepackage{enumitem}
\usepackage{mathtools}
\usepackage{tikz}
\usepackage{circuitikz}
\usepackage{verbatim}
%\usepackage{tfrupee}
\usepackage[breaklinks=true]{hyperref}
%\usepackage{stmaryrd}
\usepackage{tkz-euclide} % loads  TikZ and tkz-base
%\usetkzobj{all}
\usepackage{listings}
    \usepackage{color}                                            %%
    \usepackage{array}                                            %%
    \usepackage{longtable}                                        %%
    \usepackage{calc}                                             %%
    \usepackage{multirow}                                         %%
    \usepackage{hhline}                                           %%
    \usepackage{ifthen}                                           %%
  %optionally (for landscape tables embedded in another document): %%
    \usepackage{lscape}     
\usepackage{multicol}
\usepackage{chngcntr}
%\usepackage{enumerate}

%\usepackage{wasysym}
%\newcounter{MYtempeqncnt}
\DeclareMathOperator*{\Res}{Res}
%\renewcommand{\baselinestretch}{2}
\renewcommand\thesection{\arabic{section}}
\renewcommand\thesubsection{\thesection.\arabic{subsection}}
\renewcommand\thesubsubsection{\thesubsection.\arabic{subsubsection}}

\renewcommand\thesectiondis{\arabic{section}}
\renewcommand\thesubsectiondis{\thesectiondis.\arabic{subsection}}
\renewcommand\thesubsubsectiondis{\thesubsectiondis.\arabic{subsubsection}}

% correct bad hyphenation here
\hyphenation{op-tical net-works semi-conduc-tor}
\def\inputGnumericTable{}                                 %%

\lstset{
%language=C,
frame=single, 
breaklines=true,
columns=fullflexible
}
%\lstset{
%language=tex,
%frame=single, 
%breaklines=true
%}

\begin{document}
%


\newtheorem{theorem}{Theorem}[section]
\newtheorem{problem}{Problem}
\newtheorem{proposition}{Proposition}[section]
\newtheorem{lemma}{Lemma}[section]
\newtheorem{corollary}[theorem]{Corollary}
\newtheorem{example}{Example}[section]
\newtheorem{definition}[problem]{Definition}
%\newtheorem{thm}{Theorem}[section] 
%\newtheorem{defn}[thm]{Definition}
%\newtheorem{algorithm}{Algorithm}[section]
%\newtheorem{cor}{Corollary}
\newcommand{\BEQA}{\begin{eqnarray}}
\newcommand{\EEQA}{\end{eqnarray}}
\newcommand{\define}{\stackrel{\triangle}{=}}

\bibliographystyle{IEEEtran}
%\bibliographystyle{ieeetr}


\providecommand{\mbf}{\mathbf}
\providecommand{\pr}[1]{\ensuremath{\Pr\left(#1\right)}}
\providecommand{\qfunc}[1]{\ensuremath{Q\left(#1\right)}}
\providecommand{\sbrak}[1]{\ensuremath{{}\left[#1\right]}}
\providecommand{\lsbrak}[1]{\ensuremath{{}\left[#1\right.}}
\providecommand{\rsbrak}[1]{\ensuremath{{}\left.#1\right]}}
\providecommand{\brak}[1]{\ensuremath{\left(#1\right)}}
\providecommand{\lbrak}[1]{\ensuremath{\left(#1\right.}}
\providecommand{\rbrak}[1]{\ensuremath{\left.#1\right)}}
\providecommand{\cbrak}[1]{\ensuremath{\left\{#1\right\}}}
\providecommand{\lcbrak}[1]{\ensuremath{\left\{#1\right.}}
\providecommand{\rcbrak}[1]{\ensuremath{\left.#1\right\}}}
\theoremstyle{remark}
\newtheorem{rem}{Remark}
\newcommand{\sgn}{\mathop{\mathrm{sgn}}}
%\providecommand{\abs}[1]{\left\vert#1\right\vert}
\providecommand{\res}[1]{\Res\displaylimits_{#1}} 
%\providecommand{\norm}[1]{\left\lVert#1\right\rVert}
%\providecommand{\norm}[1]{\lVert#1\rVert}
\providecommand{\mtx}[1]{\mathbf{#1}}
%\providecommand{\mean}[1]{E\left[ #1 \right]}
\providecommand{\fourier}{\overset{\mathcal{F}}{ \rightleftharpoons}}
%\providecommand{\hilbert}{\overset{\mathcal{H}}{ \rightleftharpoons}}
\providecommand{\system}{\overset{\mathcal{H}}{ \longleftrightarrow}}
	%\newcommand{\solution}[2]{\textbf{Solution:}{#1}}
\newcommand{\solution}{\noindent \textbf{Solution: }}
\newcommand{\cosec}{\,\text{cosec}\,}
\providecommand{\dec}[2]{\ensuremath{\overset{#1}{\underset{#2}{\gtrless}}}}
\newcommand{\myvec}[1]{\ensuremath{\begin{pmatrix}#1\end{pmatrix}}}
\newcommand{\mydet}[1]{\ensuremath{\begin{vmatrix}#1\end{vmatrix}}}
%\numberwithin{equation}{section}
%\numberwithin{equation}{subsection}
%\numberwithin{problem}{section}
%\numberwithin{definition}{section}
\makeatletter
\@addtoreset{figure}{problem}
\makeatother

\let\StandardTheFigure\thefigure
\let\vec\mathbf
%\renewcommand{\thefigure}{\theproblem.\arabic{figure}}
\renewcommand{\thefigure}{\theproblem}
%\setlist[enumerate,1]{before=\renewcommand\theequation{\theenumi.\arabic{equation}}
%\counterwithin{equation}{enumi}


%\renewcommand{\theequation}{\arabic{subsection}.\arabic{equation}}

\def\putbox#1#2#3{\makebox[0in][l]{\makebox[#1][l]{}\raisebox{\baselineskip}[0in][0in]{\raisebox{#2}[0in][0in]{#3}}}}
     \def\rightbox#1{\makebox[0in][r]{#1}}
     \def\centbox#1{\makebox[0in]{#1}}
     \def\topbox#1{\raisebox{-\baselineskip}[0in][0in]{#1}}
     \def\midbox#1{\raisebox{-0.5\baselineskip}[0in][0in]{#1}}

\vspace{3cm}

\title{
Assignment - 6
}
\author{AI22MTECH02003 - Shrey Satapara}	


\maketitle


\textbf{Q-115 (Dec 2017)}
Suppose \(X_1,\dots,X_n\) are i.i.d. random vectors from \(N_p(0,\Sigma)\). Let \(l \in R^p,\:E(\Sigma_{i=1}^nl^tX_iX_i^tl)=c\) and \(E(\Sigma_{i=1}^nX_iX_i^t)=A\).\\ Which of the following statements are necessarily true?
\\\\
1.\quad\(c = l^tl\) \\\\
2.\quad\(l^t(\Sigma_{i=1}^nX_iX_i^t)l\) follows a chi-squared distribution\\\\
3.\quad\(l^t(\Sigma_{i=1}^{n_1}X_iX_i^t)l\) and \(l^t(\Sigma_{i={n_1+1}}^nX_iX_i^t)l\) are independently distributed for \(1\leq n_1\leq n-1\).\\\\
4.\quad\(A=\Sigma\)\\\\

\textbf{Solution}
Here, \(X_1,\dots,X_n\) are i.i.d. random vectors from \(N_p(0,\Sigma)\).\\\\
So, \(\Sigma_{i=1}^n(X_iX_i^t)\sim W_p(\Sigma,n)\) which is a wishart distribution with degree of freedom \(n\) and scale matrix \(\Sigma\).\\

\(W_p(\Sigma,n)\) is a probability distribution on the set of \(p x p\) symmetric non-negative definite random matrices. it is the sum of zero mean (multivariate) normal random variables squared.\\

now,
\begin{equation}
        \Sigma_{i=1}^nl^tX_iX_i^tl = \Sigma_{i=1}^n(l^tX_i)(l^tX_i)^t
\end{equation}
Let's take \(y_i = l^tX_i\) and here \(X_i \sim N(0,\Sigma)\) then \(y_i \sim N(0,l^t\Sigma l)\)  So,
\begin{equation}
\begin{split}
    \Sigma_{i=1}^nl^tX_iX_i^tl &= \Sigma_{i=1}^ny_iy_i^t
\end{split}
\end{equation}
from eq 1,2 we can say that 
\begin{equation}
\Sigma_{i=1}^nl^tX_iX_i^tl \sim (l^t\Sigma l)\chi_n^2    
\end{equation}
Hence option 2 is correct because \(l^t(\Sigma_{i=1}^{n_1}X_iX_i^t)l\) follows non-central chi square distribution\\

Option 3 is correct because we know that functions of independent random variables are independent and here \(l^t(\Sigma_{i=1}^{n_1}X_iX_i^t)l\) and \(l^t(\Sigma_{i={n_1+1}}^nX_iX_i^t)l\) are functions of \(X_i\) which are i.i.d in nature and resultant functions are also independent.\\\\
For option 1,4 first We'll find \(E(\Sigma_{i=1}^nX_iX_i^t)\)
\begin{equation}
    \begin{split}
        E(\Sigma_{i=1}^nX_iX_i^t) &= \Sigma_{i=1}^nE(X_iX_i^t)\\
        &=nE(X_iX_i^t)
    \end{split}
\end{equation}
Where the equality follows from the linearity of the expectation and that \(X_i\) are identically distributed. but for any random vector Y, \(Cov(Y) = E(YY^t) - E(Y)E(Y^t)\). Hence from eq 4
\begin{equation}
\begin{split}
    nE(X_iX_i^t) &= n(Cov(X_iX_i^t) - E(X_i)E(X_i^t))\\
    &=nCov(X_iX_i^t)\\
    &=n\Sigma
\end{split}
\end{equation}
becuase \(E(X_i) = 0\) and \(Cov(X_iX_i^t)=\Sigma\). Hence Option 4 is incorrect
Nor here from,
\begin{equation}
    \begin{split}
        E(\Sigma_{i=1}^nX_iX_i^t) = n\Sigma
    \end{split}
\end{equation}
So from eq 6,
\begin{equation}
    \begin{split}
        l^tE(\Sigma_{i=1}^nX_iX_i^t)l &= nl^t\Sigma l
        &=c
    \end{split}
\end{equation}
Hence, Option 1 is also Incorrect.\\\\
\textbf{So here, option 2,3 are correct and option 1,4 are incorrect}
\end{document}


